\section{Технико-экономическое обоснование эффективности разработки и использования программного продукта по управлению централизованными продажами в системе Bycard}

\subsection{Характеристика программного продукта}
Разработанное программное обеспечение по управлению централизованными продажами в системе Bycard предоставляет возможность использовать единый интерфейс для покупки билетов на различные мероприятия. Это даёт возможность подключения к ситеме клиентов для продажи билетов: сайты, мобильные приложения. Данная система аггрегирует имеющуюся информацию о проходящих мероприятиях, расписание, цены на билеты. Это даёт возможность выгрузки данных в автоматическом режиме, для последующего размещения на информационных ресурсах.

Большое значение имеют затраты на подключение к системе группы новых объектов, синхронизации данных между объектами.

Внедрение данного программного продукта позволят:
\begin{enumerate}
    \item хранить данные о расписании, ценах на билеты проходящих мероприятий;
    \item возможность подключения клиентов для продажи билетов;
    \item подключать новые объекты для продажи билетов через интернет;
    \item выгружать информацию о мероприятиях на информационные ресурсы в автоматическом режиме;
    \item следить за ходом продаж в режиме реального времени;
    \item поддерживать работоспособность подсистем и объектов, подключенных к системе.
\end{enumerate}

Экономическая целесообразность инвестиций в разработку и использование программного продукта осуществляется на основе расчета и оценки следующих показателей:
\begin{itemize}
    \item чистая  дисконтированная стоимость (ЧДД);
    \item срок окупаемости инвестиций (ТОК);
    \item рентабельность инвестиций (Ри).
\end{itemize}

«Разработанное программное обеспечение по управлению централизованными продажами в системе Bycard» позволяет: подключить объекты к продаже билетов через Интернет, уменьшить затраты на поддержку программного обеспечения. Тем самым увеличивая продажи, предоставляя больше возможностей и удобств для конечного пользователя.

Разработка проектов программных средств связана со значительными затратами ресурсов (трудовых, материальных, финансовых). В связи с этим создание и реализация каждого проекта программного обеспечения нуждается в соответствующем технико-экономическом обосновании (ТЭО).

Для оценки экономической эффективности инвестиционного проекта по разработке и внедрению программного продукта необходимо рассчитать:
\begin{enumerate}
    \item результат (Р), получаемый от использования программного продукта;
    \item затраты (инвестиции), необходимые для разработки программного продукта;
    \item показатели эффективности инвестиционного проекта по производству программного продукта.
\end{enumerate}

\subsection{Расчет стоимостной оценки затрат}

Общие капитальные вложения (Ко) заказчика (потребителя), связанные с приобретением, внедрением и использованием ПС, рассчитываются по формуле:

\begin{displaymath}
  K_{\text{o}} = K_{\text{пр}} + K_{\text{oс}},
\end{displaymath}

где \( K_{\text{пр}} \) --- затраты пользователя на приобретение ПС по отпускной цене разработчика с учетом стоимости услуг по эксплуатации и сопровождению (тыс.руб.);

\( K_{\text{пр}} \) --- затраты пользователя на освоение ПС (тыс. руб.).

\subsubsection{1. Расчет Затрат на разработку и отпускной цены программного продукта}

Основная заработная плата исполнителей на наш программный продукт рассчитывается по формуле:

\begin{displaymath}
  \text{З}_{\text{o}} = \sum\limits_{i=1}^n T_{\text{чi}} \cdot \text{T}_{\text{ч}} \cdot \text{Ф}{\text{n}} \cdot K,
\end{displaymath}

где n - количество исполнителей, занятых разработкой наше программного продукта;      

\( T_{\text{чi}} \) часовая тарифная ставка i-го исполнителя (тыс. руб.);

\( \text{Ф}{\text{n}} \) --- плановый фонд рабочего времени i-го исполнителя (дн.);

\( \text{T}_{\text{ч}} \) --- количество часов работы в день (ч);

К --- коэффициент премирования.

Коэффициент премирования 1,5. Для расчета заработной платы месячная тарифная ставка 1-го разряда на предприятии установлено на уровне одного миллиона ста пятидесяти тысяч рублей.


{\footnotesize
  \tablecaption{Расчет основной зработной платы}
  \label{salary-list}
  \tablefirsthead{
    \hline
    \multicolumn{1}{|>{\centering}p{0.33\textwidth}|}{\textbf{Исполнитель}} &
    \multicolumn{1}{>{\centering}p{0.07\textwidth}|}{\textbf{Разряд}} &
    \multicolumn{1}{>{\centering}p{0.11\textwidth}|}{\textbf{Тарифный коэффициент}} &
    \multicolumn{1}{>{\centering}p{0.11\textwidth}|}{\textbf{Месячная тарифная ставка, руб.}} &
    \multicolumn{1}{>{\centering}p{0.09\textwidth}|}{\textbf{Часовая тарифная ставка, руб.}} &
    \multicolumn{1}{>{\centering}p{0.12\textwidth}|}{\textbf{Заработная плата, руб.}}\\
  }
  \tablehead{
    \multicolumn{6}{c}{{\normalsize Продолжение таблицы \thetable{}}}\\
    \hline
    \multicolumn{1}{|>{\centering}p{0.33\textwidth}|}{\textbf{Исполнитель}} &
    \multicolumn{1}{>{\centering}p{0.07\textwidth}|}{\textbf{Разряд}} &
    \multicolumn{1}{>{\centering}p{0.11\textwidth}|}{\textbf{Тарифный коэффициент}} &
    \multicolumn{1}{>{\centering}p{0.11\textwidth}|}{\textbf{Месячная тарифная ставка, руб.}} &
    \multicolumn{1}{>{\centering}p{0.09\textwidth}|}{\textbf{Часовая тарифная ставка, руб.}} &
    \multicolumn{1}{>{\centering}p{0.12\textwidth}|}{\textbf{Заработная плата, руб.}}\\
  }
  \begin{xtabular}{|l|c|c|c|c|c|}
    \hline
    \multicolumn{1}{|>{}p{0.33\textwidth}|}{программист II категории} & \( 12 \) & \( 1,54 \) & \( 11,100 \)  & \( 40 \) & \( 5328,000 \) \\
    \hline
    \multicolumn{1}{|>{}p{0.33\textwidth}|}{ведущий программист} & \( 15 \) & \( 1,7 \) & \( 12,250 \)  & \( 30 \) & \( 4410,000 \) \\
    \hline
    \multicolumn{1}{|>{}p{0.33\textwidth}|}{Начальник, руководитель проекта} & \( 16 \) & \( 1,96 \) & \( 14,100 \)  & \( 30 \) & \( 5076,000 \) \\
    \hline
    \multicolumn{1}{|>{}p{0.33\textwidth}|}{Итого с премией (50\%), \( \text{З}_{\text{o}} \) } & - & - & - & - & \( 14814,000 \) \\
    \hline
  \end{xtabular}
}\\

Дополнительная заработная плата на наш программный продукт \( \text{З}_{\text{д}} \) включает выплаты, предусмотренные законодательством о труде (оплата отпусков, льготных часов, времени выполнения государственных обязанностей и других выплат, не связанных с основной деятельностью исполнителей), и определяется по нормативу в процентах к основной заработной плате:

\begin{displaymath}
  \text{З}_{\text{д}} = \frac{ \text{З}_{\text{о}} \cdot \text{Н}_{\text{д}} } { 100 },
\end{displaymath}

где \(\text{З}_{\text{д}}\) --- дополнительная заработная плата исполнителей (тыс. руб.);
 
\(\text{Н}_{\text{д}}\) --- норматив дополнительной заработной платы равный 10\%.

\begin{displaymath}
  \text{З}_{\text{д}} = \frac{ 14814,000 \cdot 10 } { 100 } = 1481,400 \text{ (тыс. руб.)},
\end{displaymath}

Отчисления в фонд социальной защиты населения и обязтельное страхование \( \text{З}_{\text{сз}} \) определяются в соответствии с действующими законодательными актами по нормативу в процентном отношении к фонду основной и дополнительной зарплаты исполнителей, определенной по нормативу, установленному в целом по организации:

\begin{displaymath}
  \text{З}_{\text{сз}} = \frac{ (\text{З}_{\text{о}} + \text{З}_{\text{д}}) \cdot \text{Н}_{\text{сх}}  } { 100},
\end{displaymath}

где \(\text{Н}_{\text{сз}}\) --- норматив отчислений в фонд социальной защиты населения  и на обязательное страхование (34 + 0,6\%).

\begin{displaymath}
  \text{З}_{\text{сз}} = \frac{ (14814,000 + 1481,400) \cdot 34,6  } { 100 } = 5638,208 \text{ (тыс. руб.)}.
\end{displaymath}

Расходы по статье «Машинное время» (Рм) включают оплату машинного времени, необходимого для разработки и отладки программного продукта, которое определяется по нормативам (в машино-часах) на 100 строк исходного кода (Hмв) машинного времени, и определяются по формуле:

\begin{displaymath}
  \text{Р}_{\text{м}} = \text{Ц}_{\text{м}} \cdot \text{Т}_{\text{пр}},
\end{displaymath}

где \(\text{Ц}_{\text{м}}\) --- цена одного машино-часа. Рыночная стоимость машино-часа компьютера со всеми необходимым оборудованием (12 тыс. руб. / ч);

\(\text{Т}_{\text{пр}}\) --- время работы над программным продуктом (100 ч).

\begin{displaymath}
  \text{Р}_{\text{м}} = 12 \cdot 100 = 1200 \text{ (тыс. руб.)}.
\end{displaymath}

Расходы по статье «Научные командировки» (\(\text{Р}_{\text{нк}}\)) на програмнное средство определяются по формуле:

\begin{displaymath}
  \text{Р}_{\text{нк}} = \frac{\text{З}_{\text{о}} \cdot \text{Н}_{\text{рик}}}{100},
\end{displaymath}

где \(\text{Н}_{\text{рик}}\) --- норматив расходов на командировки в целом по организации (\%). Норматив на командировки - 10 \% от основной заработной платы.

\begin{displaymath}
  \text{Р}_{\text{нк}} = \frac{14814,000 \cdot 10}{100} = 1481,400 \text{ (тыс. руб.)}.
\end{displaymath}

Расходы по статье «Прочие затраты» (\(\text{П}_{\text{з}}\)) на программное средство включают затраты на приобретение и подготовку специальной научно-технической информации и специальной литературы. И определяются по формуле:

\begin{displaymath}
  \text{П}_{\text{з}} = \frac{\text{З}_{\text{о}} \cdot \text{Н}_{\text{пз}}}{100},
\end{displaymath}

где (\(\text{Н}_{\text{пз}}\)) --- норматив прочих затрат в целом по организации равен 20\%

\begin{displaymath}
  \text{П}_{\text{з}} = \frac{14814,000 \cdot 20}{100} = 2961,800 \text{ (тыс. руб.)}.
\end{displaymath}

Затраты по статье «Накладные расходы» ((\(\text{Р}_{\text{н}}\))), связанные с необходимостью содержания аппарата управления, вспомогательных хозяйств и опытных (экспериментальных) производств, а также с расходами на общехозяйственные нужды, и определяют по формуле:

\begin{displaymath}
  \text{Р}_{\text{н}} = \frac{\text{З}_{\text{о}} \cdot \text{Н}_{\text{рн}}}{100},
\end{displaymath}

где \(\text{Р}_{\text{н}}\) --- накладные расходы на программный продукт (тыс. руб.);

\(\text{Н}_{\text{рн}}\) --- норматив накладных расходов в целом по организации,100\%.

\begin{displaymath}
  \text{Р}_{\text{н}} = \frac{14814,000 \cdot 100}{100} = 14814,000 \text{ (тыс. руб.)}.
\end{displaymath}

Общая сумма расходов по смете (\(\text{С}_{\text{р}}\)) на программный продукт рассчитывается по формуле:

\begin{displaymath}
  \text{С}_{\text{р}} = \text{З}_{\text{о}} + \text{З}_{\text{д}} + \text{З}_{\text{сз}} + \text{Р}_{\text{м}} + \text{Р}_{\text{нк}} + \text{П}_{\text{з}} + \text{Р}_{\text{н}}
\end{displaymath}

\begin{displaymath}
  \text{С}_{\text{р}} = 14814,000+1481,400+5638,208+500+1481,400+
\end{displaymath}
\begin{displaymath}
  +2962,800+14814,000=41691.808 \text{ (тыс. руб.)}.
\end{displaymath}


Кроме того, организация-разработчик осуществляет затраты на сопровождение и адаптацию программного продукта \(\text{Р}_{\text{са}}\), которые определяются по формуле:

\begin{displaymath}
  \text{Р}_{\text{са}} = \frac{\text{С}_{\text{р}} \cdot \text{Н}_{\text{рса}}}{100},
\end{displaymath}

где \(\text{Н}_{\text{рса}}\) --- норматив расходов на сопровождение и адаптацию 10\%.

\begin{displaymath}
  \text{Р}_{\text{са}} = \frac{41691,808 \cdot 10}{100} = 4169,181 \text{ (тыс. руб.)}.
\end{displaymath}

Общая сумма расходов на разработку (с затратами на сопровождение и адаптацию) как полная себестоимость программно продукта (\(\text{С}_{\text{п}}\)) определяется по формуле:

\begin{displaymath}
  \text{С}_{\text{п}} = \text{С}_{\text{р}} + \text{Р}_{\text{са}},
\end{displaymath}

\begin{displaymath}
  \text{Р}_{\text{са}} = 41691,808+4169,181 = 45860,989 \text{ (тыс. руб.)}.
\end{displaymath}

Прибыль рассчитывается по формуле:

\begin{displaymath}
  \text{П}_{\text{о}} = \frac{\text{С}_{\text{п}} \cdot \text{У}_{\text{рп}}}{100},
\end{displaymath}

где \(\text{П}_{\text{о}}\) --- прибыль от реализации программного продукта заказчику (тыс. руб.);

\(\text{У}_{\text{рп}}\) --- уровень рентабельности программного продукта 25\%; 

\(\text{С}_{\text{п}}\) --- себестоимость програмнного продукта (тыс. руб.).

\begin{displaymath}
  \text{П}_{\text{о}} = \frac{45860,989 \cdot 25}{100} = 11465,247 \text{ (тыс. руб.)}.
\end{displaymath}

Прогнозируемая цена нашего программного продукта без налогов (Цп): 

\begin{displaymath}
  \text{Ц}_{\text{п}} = \text{С}_{\text{р}} + \text{П}_{\text{о}},
\end{displaymath}

\begin{displaymath}
  \text{Ц}_{\text{п}} = 41691,808+11465,247 = 53157,055 \text{ (тыс. руб.)}.
\end{displaymath}

\subsection{Расчет стоимостной оценки результата}

Результатом (Р) в сфере использования нашего программного продукта является прирост чистой прибыли и амортизационных отчислений.

\subsubsection{Расчет прироста чистой прибыли}

Прирост чистой прибыли представляет собой экономию затрат на заработную плату и начислений на заработную плату, полученную в результате внедрения программного продукта, составит:

\begin{displaymath}
  \text{Э}_{\text{з}} = \text{К}_{\text{пр}}\cdot(\text{t}_{\text{c}}\cdot\text{T}_{\text{c}}-\text{t}_{\text{н}}\cdot\text{T}_{\text{н}})\cdot\text{N}_{\text{n}}\cdot
  (1+\frac{\text{Н}_{\text{дп}}}{100})\cdot(1+\frac{\text{Н}_{\text{нпо}}}{100}),
\end{displaymath}

где \(\text{Н}_{\text{п}}\) --- плановый объем работ по анализу и обработки результатов, сколько раз выполнялись в году;

\(\text{t}_{\text{c}}\) --- трудоемкость выполнения работы до внедрения программного продукта; 

\(\text{t}_{\text{c}}\) --- трудеемкость выполнения работы после вднедрения програмнного продукта;

\(\text{T}_{\text{c}}\) --- часовая тарифная ставка, соответсвующая разряду выполеняемых работ до внедрения программного продукта;

\(\text{T}_{\text{n}}\) --- часовая тарифная ставка, соответсвующая разряду выполеняемых работ после внедрения программного продукта;

\(\text{К}_{\text{пр}}\) --- коэффициент премий 1.5; 

\(\text{Н}_{\text{д}}\) --- номратив дополнительной заработной платы 20\%;

\(\text{Н}_{\text{по}}\) --- ставка отчислений в ФСЗН и обязательное страхование 34+0,6\%. 

\begin{displaymath}
    \text{Э}_{\text{з}} = 1,5\cdot(160\cdot10-20\cdot10)\cdot12\cdot(1+\frac{20}{100})\cdot(1+\frac{34,6}{100}) = 42621,898 \text{ (тыс. руб.)}.
\end{displaymath}

Прирост чистой прибыли рассчитывается по формуле:

\begin{displaymath}
  \text{П}_{\text{ч}} = \sum\limits_{i=1}^n \text{Э}_{\text{i}}\cdot(1-\frac{\text{Н}_{\text{п}}}{100}),
\end{displaymath}

где \(\text{n}\) --- виды затрат, по которым получена экономия;

\(\text{Н}_{\text{п}}\) --- ставка налога на прибыль, 18\%.

\begin{displaymath}
    \text{П}_{\text{ч}} = 41865,984\cdot(1-\frac{18}{100}) = 34330,107 \text{ (тыс. руб.)}.
\end{displaymath}

\subsubsection{Расчет прироста амортизационных отчислений}

Амортизационные отчисления являются источником погашения инвестиций в приобретение программного продукта. Расчет амортизационных отчислений осуществляется по формуле:

\begin{displaymath}
  \text{А} = \frac{\text{Н}_{\text{а}}\cdot\text{И}_{\text{об}}}{100},
\end{displaymath}

где \(\text{Н}_{\text{а}}\) --- норма амортизации программного продукта 20\%;
\(\text{И}_{\text{об}}\) --- стоимость программного продукта, тыс. руб.

\begin{displaymath}
  \text{А} = \frac{20\cdot53157,055}{100} = 8759,107 \text{ (тыс. руб.)}.
\end{displaymath}

\subsection{Расчет показателей экономической эффективности проекта}

При оценке эффективности инвестиционных проектов необходимо осуществить приведение затрат и результатов, полученных в разные периоды времени, к  расчетному году,  путем умножения затрат и результатов на коэффициент дисконтирования \(\alpha_{\text{t}}\), который определяется следующим образом:

\begin{displaymath}
  \alpha_{\text{t}} = \frac{1}{(1+\text{Е})^{t-\text{t}_{\text{p}}}},
\end{displaymath}

где \(\text{Е}_{\text{н}}\) --- требуемая норма дисконта;  

\(\text{t}\) --- порядковый номер года, затраты и результаты которого приводятся к расчетному году;

\(\text{t}_{\text{p}}\) --- расчетный год, в качестве расчетного года принимается год вложения инвестиций, равный 1.

\begin{displaymath}
  \alpha_{\text{1}} = \frac{1}{(1+0.33)^{1-1}} = 1;
\end{displaymath}

\begin{displaymath}
  \alpha_{\text{1}} = \frac{1}{(1+0.33)^{2-1}} = 0,75;
\end{displaymath}

\begin{displaymath}
  \alpha_{\text{1}} = \frac{1}{(1+0.33)^{3-1}} = 0,57;
\end{displaymath}

\begin{displaymath}
  \alpha_{\text{1}} = \frac{1}{(1+0.33)^{4-1}} = 0,43.
\end{displaymath}

Расчет чистого дисконтированного дохода за четыре года реализации проекта и срока окупаемости инвестиций представлены в таблице таблице ~\ref{econom-list}:

{\footnotesize
  \tablecaption{Экономические результаты работы предприятия}
  \label{econom-list}
  \tablefirsthead{
    \hline
    \multirow{2}{0.2\textwidth}{\centering \textbf{Наименование показателей}} &
    \multirow{2}{0.08\textwidth}{\centering \textbf{Един. измер.}} &
    \multirow{2}{0.08\textwidth}{\centering \textbf{Усл. обоз.}} &
    \multicolumn{4}{>{\centering}p{0.45\textwidth}|}{\textbf{По годам производства}} \\
    \cline{4-7}
    & & & 
    \multicolumn{1}{>{\centering}p{0.1125\textwidth}|}{\textbf{1-й}} &
    \multicolumn{1}{>{\centering}p{0.1125\textwidth}|}{\textbf{2-й}} &
    \multicolumn{1}{>{\centering}p{0.1125\textwidth}|}{\textbf{3-й}} &
    \multicolumn{1}{>{\centering}p{0.1125\textwidth}|}{\textbf{4-й}} \\
  }
  \tablehead{
    \multicolumn{4}{c}{{\normalsize Продолжение таблицы \thetable{}}}\\
    \hline
    \multirow{2}{0.2\textwidth}{\centering \textbf{Наименование показателей}} &
    \multirow{2}{0.08\textwidth}{\centering \textbf{Един. измер.}} &
    \multirow{2}{0.08\textwidth}{\centering \textbf{Усл. обоз.}} &
    \multicolumn{4}{>{\centering}p{0.45\textwidth}|}{\textbf{По годам производства}} \\
    \cline{4-7}
    & & & 
    \multicolumn{1}{>{\centering}p{0.1125\textwidth}|}{\textbf{1-й}} &
    \multicolumn{1}{>{\centering}p{0.1125\textwidth}|}{\textbf{2-й}} &
    \multicolumn{1}{>{\centering}p{0.1125\textwidth}|}{\textbf{3-й}} &
    \multicolumn{1}{>{\centering}p{0.1125\textwidth}|}{\textbf{4-й}} \\
  }
  \begin{xtabular}{|l|c|c|c|c|c|c|}
    \hline
    \textbf{Результат} & & & & & & \\
    \hline
    \multicolumn{1}{|>{}p{0.2\textwidth}|}{1. Прирост чистой прибыли} & \( \text{тыс. руб.} \) & \( \Delta\text{П}_\text{ч} \) & \( 17165,0535 \) & \( 34330,107 \) & \( 34330,107 \) & \( 34330,107 \) \\
    \hline
    \multicolumn{1}{|>{}p{0.2\textwidth}|}{2. Прирост амортизационных отчислений} & \( \text{тыс. руб.} \) & \( \Delta\text{А} \) & \( 5315,706 \) & \( 10631,411 \) & \( 10631,411 \) & \( 10631,411 \) \\
    \hline
    \multicolumn{1}{|>{}p{0.2\textwidth}|}{3. Прирост результата} & \( \text{тыс. руб.} \) & \( \Delta\text{P}_\text{t} \) & \( 22480,759 \) & \( 44961,518 \) & \( 44961,518 \) & \( 44961,518 \) \\
    \hline
    \multicolumn{1}{|>{}p{0.2\textwidth}|}{4. Коэффициент дисконтирования} & \( \text{тыс. руб.} \) & \( \alpha_t \) & \( 1 \) & \( 0,813 \) & \( 0,661 \) & \( 0,537 \) \\
    \hline
    \multicolumn{1}{|>{}p{0.2\textwidth}|}{5. Результат с учетом фактора времени} & \( \text{тыс. руб.} \) & \( \text{P}_\text{t} \cdot \alpha_t \) & \( 22480,759 \) & \( 41772,0947 \) & \( 39539,498 \) & \( 37944,787 \) \\
    \hline
    \multicolumn{1}{|>{}p{0.2\textwidth}|}{\textbf{Затраты (инвестиции)}}
     & & & & & & \\
    \hline
    \multicolumn{1}{|>{}p{0.2\textwidth}|}{6. Инвестиции в разработку программного продукта} & \( \text{тыс. руб.} \) & \( \text{И}_\text{об} \) & \( 53157,055 \) & - & - & -\\
    \hline
    \multicolumn{1}{|>{}p{0.2\textwidth}|}{7. Инвестиции с учетом фактора времени} & \( \text{тыс. руб.} \) & \( \text{И}_\text{t} \cdot \alpha_t \) & \( 53157,055 \) & - & - & -\\
    \hline
    \multicolumn{1}{|>{}p{0.2\textwidth}|}{8. Чистый дисконтированный доход по годам (п.4 - п.6)} & \( \text{тыс. руб.} \) & \( \text{ЧДД}_\text{t} \) & \( -30676,296 \) & \( 11095,799 \) & \( 39539,498 \) & \( 37944,787 \) \\
    \hline
    \multicolumn{1}{|>{}p{0.2\textwidth}|}{9. ЧДД нарастающим итогом} & \( \text{тыс. руб.} \) & \( \text{ЧДД} \) & \( -30676,296 \) & \( 11095,799 \) & \( 50635,297 \) & \( 88580,084 \) \\
    \hline
  \end{xtabular}
}\\

Рассчитаем рентабельность инвестиций (\(\text{Р}_{\text{и}}\)) по формуле:

\begin{displaymath}
  \text{Р}_{\text{и}} = \frac{\text{П}_{\text{чср}}}{\text{З}}\cdot100,
\end{displaymath}

где \(\text{З}\) --- затраты на приобретения нашего программного продукта;

\(\text{П}_{\text{чср}}\) --- среднегодовая величина чистой прибыли за расчетный период, тыс. руб., которая определяется по формуле:

\begin{displaymath}
  \text{П}_{\text{чср}} = \frac{\sum\limits_{i=1}^n \text{П}_{\text{чт}}}{n},
\end{displaymath}

где \(\text{П}_{\text{чт}}\) --- чистая прибыль, полученная в году t, тыс. руб. 

\begin{displaymath}
  \text{П}_{\text{чср}} = \frac{-30676,296+41772,095+39539,498+37944,787}{4} =
\end{displaymath}
\begin{displaymath}
  = 22145,021 \text{ (тыс. руб.)}.
\end{displaymath}

\begin{displaymath}
  \text{P}_{\text{u}} = \frac{22145,021}{53157,055}\cdot100=42(\%).
\end{displaymath}

В результате технико-экономического обоснования инвестиций по производству нового изделия были получены следующие значения показателей их эффективности:
\begin{enumerate}
    \item чистый дисконтированный доход за четыре года производства продукции составит тыс. руб.;
    \item все инвестиции окупаются на 2 год;  
    \item рентабельность инвестиций составляет 42\%.
\end{enumerate}

Таким образом, внедрение программного продукта «Программное обеспечение по управлению централизованными продажами в системе Bycard» являетcя эффективным и инвестиции в его разработку целесообразны.