\section{Технико-экономическое обоснование эффективности разработки и использования программного продукта по управлению централизованными продажами в системе bycard}

\subsection{Характеристика программного продукта}
Разработанное программное обеспечение по управлению централизованными продажами в системе Bycard предоставляет возможность использовать единый интерфейс для покупки билетов на различные мероприятия. Это даёт возможность подключения к ситеме клиентов для продажи билетов: сайты, мобильные приложения. Данная система аггрегирует имеющуюся информацию о проходящих мероприятиях, расписание, цены на билеты. Это даёт возможность выгрузки данных в автоматическом режиме, для последующего размещения на информационных ресурсах.

Большое значение имеют затраты на подключение к системе группы новых объектов, синхронизации данных между объектами.

Внедрение данного программного продукта позволят:
\begin{enumerate}
    \item хранить данные о расписании, ценах на билеты проходящих мероприятий;
    \item возможность подключения клиентов для продажи билетов;
    \item подключать новые объекты для продажи билетов через интернет;
    \item выгружать информацию о мероприятиях на информационные ресурсы в автоматическом режиме;
    \item следить за ходом продаж в режиме реального времени;
    \item поддерживать работоспособность подсистем и объектов, подключенных к системе.
\end{enumerate}

Экономическая целесообразность инвестиций в разработку и использование программного продукта осуществляется на основе расчета и оценки следующих показателей:
\begin{itemize}
    \item чистая  дисконтированная стоимость (ЧДД);
    \item срок окупаемости инвестиций (ТОК);
    \item рентабельность инвестиций (Ри).
\end{itemize}

«Разработанное программное обеспечение по управлению централизованными продажами в системе Bycard» позволяет: подключить объекты к продаже билетов через Интернет, уменьшить затраты на поддержку программного обеспечения. Тем самым увеличивая продажи, предоставляя больше возможностей и удобств для конечного пользователя.

Разработка проектов программных средств связана со значительными затратами ресурсов (трудовых, материальных, финансовых). В связи с этим создание и реализация каждого проекта программного обеспечения нуждается в соответствующем технико-экономическом обосновании (ТЭО).

Для оценки экономической эффективности инвестиционного проекта по разработке и внедрению программного продукта необходимо рассчитать:
\begin{enumerate}
    \item результат (Р), получаемый от использования программного продукта;
    \item затраты (инвестиции), необходимые для разработки программного продукта;
    \item показатели эффективности инвестиционного проекта по производству программного продукта.
\end{enumerate}

\subsection{Расчет экономического эффекта у разработчика}
Проект ``Методы распознавания, захвата и сопровождения движущихся объектов и их применение в задаче отслеживания людей''. Среда разработки --- GNU Emacs, ПО общего назначения. \( V_i = 21670 LOC \).

{\footnotesize
  \tablecaption{Перечень и объем функций программного модуля}
  \label{function-list}
  \tablefirsthead{\hline
    \multicolumn{1}{|>{\centering}p{0.12\textwidth}|}{\textbf{№ функции}} &
    \multicolumn{1}{>{\centering}p{0.49\textwidth}|}{\textbf{Наименование (содержание)}} &
    \multicolumn{2}{>{\centering}p{0.3\textwidth}|}{\textbf{Объем функции (LOC)}}\\
    \hline
    & & \multicolumn{1}{>{\centering}p{0.135\textwidth}|}{\textbf{по каталогу $V_i$}} &
    \multicolumn{1}{>{\centering}p{0.135\textwidth}|}{\textbf{уточненный $V_{\text{у}i}$}}\\
  }
  \tablehead{
    \multicolumn{4}{c}{{\normalsize Продолжение таблицы \thetable{}}}\\
    \hline
    \multicolumn{1}{|>{\centering}p{0.12\textwidth}|}{\textbf{№ функции}} &
    \multicolumn{1}{>{\centering}p{0.49\textwidth}|}{\textbf{Наименование (содержание)}} &
    \multicolumn{2}{>{\centering}p{0.3\textwidth}|}{\textbf{Объем функции (LOC)}}\\
    \hline
    & & \multicolumn{1}{>{\centering}p{0.135\textwidth}|}{\textbf{по каталогу $V_i$}} &
    \multicolumn{1}{>{\centering}p{0.135\textwidth}|}{\textbf{уточненный $V_{\text{у}i}$}}\\
  }
  \begin{xtabular}{|>{\centering}p{0.12\textwidth}|p{0.49\textwidth}|>{\centering}p{0.135\textwidth}|>{\centering}p{0.135\textwidth}|}
    \hline
    101 & Организация ввода информации & 150 & 50\tabularnewline
    \hline
    102 & Контроль, предварительная обработка информации & 450 & 150\tabularnewline
    \hline
    109 & Организация ввода/вывода информации в интерактивном режиме & 320 & 110\tabularnewline
    \hline
    111 & Управление вводом/выводом & 2400 & 320\tabularnewline
    \hline
    304 & Обработка файлов & 720 & 180\tabularnewline
    \hline
    309 & Формирование файла & 1020 & 190\tabularnewline
    \hline
    401 & Генерация рабочих программ & 3360 & 330\tabularnewline
    \hline
    506 & Обработка ошибочных и сбойных ситуаций & 410 & 190\tabularnewline
    \hline
    507 & Обеспечение интерфейса между компонентами & 970 & 130\tabularnewline
    \hline
    702 & Ввод информации с видеокамеры & 1380 & 130\tabularnewline
    \hline
    707 & Графический вывод результатов & 480 & 60\tabularnewline
    \hline
    801 & Реализация алгоритма AdaBoost & 1520 & 240\tabularnewline
    \hline
    802 & Реализация Context Shape & 2500 & 250\tabularnewline
    \hline
    803 & Реализация алгоритмов для работы с Pictorial Structures & 5130 & 270\tabularnewline
    \hline
    804 & Реализация венгерского метода & 720 & 130\tabularnewline
    \hline
    805 & Реализация алгортима sum-product & 140 & 150\tabularnewline
    \hline
    & Итого & 21670 & 2880\tabularnewline
    \hline
  \end{xtabular}
}\\[0.5\baselineskip]

За счет использования программных модулей с открытым исходным кодом объемы всех функций были уменьшены и уточненный объем (\( V_{\text{у}} \)) составил \( 2880 LOC \) вместо \( V_i = 21670 LOC \) (табл.~\ref{function-list}).

ПО отнесено к первой категории сложности: предполагается использовать в режиме реального времени и обеспечить существенное распараллеливание вычислений. Наличие двух характеристик, определяющих сложность ПО, позволяет применить к объему ПО коэффициент \( K_{\text{с}} \):
\begin{displaymath}
  K_{\text{с}} = 1 + 0,12 = 1,12.
\end{displaymath}

Коэффициент использования стандартных модулей --- \( K_{\text{т}} = 0,6 \), так как степень охвата реализуемых функций разрабатываемого ПО стандартными модулями составляет свыше \( 60\% \); а коэффициент новизны ПО --- \( K_{\text{н}} = 0,7 \).

Нормативная трудоемкость разработки ПО (\( T_{\text{н}} \)) составляет \( 86 ^{\text{чел.}}/_{\text{дн}} \).

Общая трудоемкость определяется как:
\begin{displaymath}
  T_{\text{о}} = T_{\text{н}} \cdot K_{\text{с}} \cdot K_{\text{т}} \cdot K_{\text{н}} = 86 \cdot 1,12 \cdot 0,6 \cdot 0,7 = 41 ^{\text{чел.}}/_{\text{дн}}.
\end{displaymath}

В соответствии с договором исполнителя с заказчиком срок разработки установлен \( 2 \text{мес} \) (\( 0,17 \text{г.} \)). Эффективный фонд времени одного работника --- \( 236 \text{дн.} \) Рассчитаем число исполнителей проекта:
\begin{displaymath}
  \text{Ч}_{\text{р}} = \frac{41}{0,17 \cdot 236} = 1 \text{чел}.
\end{displaymath}

Основной статьей расходов на создание ПО является заработная плата разработчиков (исполнителей) проекта. Коэффициент премирования --- \( 1,5 \). В соответствии со штатным расписанием на разработке будет занят ведущий программист (тарифный разряд --- \( 15 \); тарифный коэффициент --- \( 3,48 \); плановый фонд рабочего времени --- \( 41 \text{дн} \)).

Расчет часовой тарифной ставки:
\begin{displaymath}
  T_{\text{ч}} = \frac{81000 \cdot 3,48}{170} = 1660 \text{руб}.
\end{displaymath}

Расчет основной заработной платы:
\begin{displaymath}
  \text{З}_{\text{ои}} = 1,66 \cdot 8 \cdot 41 \cdot 1,5 = 816000 \text{руб}.
\end{displaymath}

Дополнительная заработная плата на конкретное ПО (\( \text{З}_{\text{ди}} \)) включает выплаты, предусмотренные законодательством о труде (оплата отпусков, льготных часов, времени выполнения государственных обязанностей и других выплат, не связанных с основной деятельностью исполнителей), и определяется по нормативу в процентах к основной заработной плате:
\begin{displaymath}
  \text{З}_{\text{ди}} = \frac{\text{З}_{\text{ои}} \cdot \text{Н}_{\text{д}}}{100} = \frac{810000 \cdot 20\%}{100\%} = 162000 \text{руб.}
\end{displaymath}

\( \text{Н}_{\text{д}} \) --- норматив дополнительной заработной платы рассчитывается по формуле
\begin{displaymath}
  \text{Н}_{\text{д}} = \frac{\text{З}_{\text{д}}}{\text{З}_{\text{о}}} \cdot 100,
\end{displaymath}
где \( \text{З}_{\text{д}} \) --- дополнительная заработная плата в целом по научной организации; \( \text{З}_{\text{о}} \) --- основная заработная плата в целом по организации.

Отчисления в фонд социальной защиты населения (\( \text{З}_{\text{сзи}} \)) определяются в соответствии с действующими законодательными актами по нормативу в процентном отношении к фонду основной и дополнительной зарплат исполнителей, определенном по нормативу, установленному в целом по организации:
\begin{displaymath}
  \text{З}_{\text{сзи}} = \frac{(\text{З}_{\text{ои}} + \text{З}_{\text{ди}}) \cdot \text{Н}_{\text{сз}}}{100\%} = \frac{(816000 + 162000) \cdot 35\%}{100\%} = 343000 \text{руб}.
\end{displaymath}

Расходы по статье ``Материалы'' определяются на основании сметы затрат, разрабатываемой на ПО с учетом действующих номартивов. По статье ``Материалы'' отражаются расходы на магнитные носители, бумагу, красящие ленты и другие материалы, необходимые для разработки ПО. Нормы расхода материалов в суммарном выражении (\( \text{Н}_{\text{м}} \)) определяются по нормативу в процентах к фонду основной заработной платы разработчиков (\( \text{Н}_{\text{мз}} \)), который устанавливается организацией.
\begin{displaymath}
  \text{М}_{\text{и}} = \frac{\text{З}_{\text{ои}} \cdot \text{Н}_{\text{мз}}}{100\%} = \frac{816000 \cdot 3\%}{100\%} = 25000 \text{руб},
\end{displaymath}
где \( \text{Н}_{\text{мз}} \) --- норма расхода материалов от основной заработной платы (\( \% \)).

Расходы по статье ``Спецоборудование'' включает затраты средств на приобретение вспомогательных, специального назначения, технических и программных средств, необходимых для разработки конкретного ПО, включая расходы на их проектирование, изготовление, отладку, установку и эксплуатацию.
\begin{displaymath}
  \text{Р}_{\text{си}} = 0 \text{руб},
\end{displaymath}
так как при разработке будут использоваться свободные программные средства с открытым исходным кодом.

Расходы по статье ``Машинное время'' включают оплату машинного времени, необходимого для разработки и отладки ПО, которое определяется по нормативам (в машино-часах) на \( 100 \) строк исходного кода (\( \text{Н}_{\text{мв}} \)) машинного времени в зависимости от характера решаемых задач и типа ПК.

Рыночная цена одного машино-часа компьютера со всем необходимым оборудованием --- \( 3600 \text{руб} \).
\begin{displaymath}
  \text{Р}_{\text{ми}} = \text{Ц}_{\text{ми}} \cdot \frac{\text{V}_{\text{ои}}}{100} \cdot \text{Н}_{\text{мв}} = 3600 \cdot \frac{2880}{100} \cdot 12 = 1244160 \text{руб}.
\end{displaymath}

Расходы по статье ``Научные командировки''. Норматив на командировки --- \( 30\% \) от основной заработной платы.
\begin{displaymath}
  \text{Р}_{\text{нки}} = \frac{\text{З}_{\text{ои}} \cdot \text{Н}_{\text{рнк}}}{100\%} = \frac{816000 \cdot 30\%}{100\%} = 243000 \text{руб}.
\end{displaymath}

Расходы по статье ``Прочие затраты'' на конкретное ПО включают затраты на приобретение и подготовку научно-технической информации и специальной литературы.
\begin{displaymath}
  \text{П}_{\text{зи}} = \frac{\text{З}_{\text{ои}} \cdot \text{Н}_{\text{пз}}}{100\%} = \frac{816000 \cdot 20\%}{100\%} = 162000 \text{руб},
\end{displaymath}
где \( \text{Н}_{\text{пз}} \) --- норматив прочих затрат в целом по организации.

Затраты по статье ``Накладные расходы'', связанные с необходимостью содержания аппарата управления, вспомогательных хозяйств и опытных производств, а также с расходами на общехозяйственные нужды (\( \text{Р}_{\text{ни}} \)), относятся на конкретное ПО по нормативу (\( \text{Н}_{\text{рн}} \)) в процентном отношении к основной заработной плате исполнителей. Норматив устанавливается в целом по организации:
\begin{displaymath}
  \text{Р}_{\text{ни}} = \frac{\text{З}_{\text{ои}} \cdot \text{Н}_{\text{рн}}}{100\%} = \frac{816000 \cdot 100\%}{100\%} = 816000 \text{руб},
\end{displaymath}
где \( \text{Р}_{\text{ни}} \) --- накладные расходы на конкретною ПО (д.е.); \( \text{Н}_{\text{рн}} \) --- норматив накладных расходов в целом по организации.

Общая сумма расходов по смете (\( \text{С}_{\text{пи}} \)) на ПО рассчитывается по формуле
\begin{align*}
  \text{С}_{\text{пи}} &= \text{З}_{\text{ои}} + \text{З}_{\text{ди}} + \text{З}_{\text{сзи}} + \text{М}_{\text{и}} + \text{Р}_{\text{си}} + \text{Р}_{\text{ми}} + \text{Р}_{\text{нки}} + \text{П}_{\text{зи}} + \text{Р}_{\text{ни}} =\\
  &= 816000 + 162000 + 343000 + 25000 + 0 + 1244160 +\\
  &+ 243000 + 162000 + 816000 = 3851160 \text{руб}.
\end{align*}

Рентабельность и прибыль по создаваемому ПО (\( \text{П}_{\text{си}} \)) oпpeдeляютcя исходя из результатов анализа рыночных условий, переговоров с заказчиком (потребителем) и согласования с ним отпускной цены, включающей дополнительно налог на добавленную стоимость и отчисления в местный и республиканский бюджеты.

Прибыль рассчитывается по формуле
\begin{displaymath}
  \text{П}_{\text{ои}} = \frac{\text{С}_{\text{пи}} \cdot \text{У}_{\text{рпи}}}{100\%} = \frac{3851160 \cdot 40\%}{100\%} = 1540464 \text{руб},
\end{displaymath}
где \( \text{П}_{\text{ои}} \) --- прибыль от реализации ПО заказчику; \( \text{У}_{\text{рпи}} \) --- уровень рентабельности ПО; \( \text{С}_{\text{пи}} \) --- себестоимость ПО.

Прогнозируемая цена ПО без налогов
\begin{displaymath}
  \text{Ц}_{\text{пи}} = \text{С}_{\text{пи}} + \text{П}_{\text{ои}} = 3851160 + 1540464 = 5391624 \text{руб},
\end{displaymath}

Налог на добавленную стоимость (\( \text{НДС}_{\text{и}} \)):
\begin{align*}
  \text{НДС}_{\text{и}} &= \frac{(\text{Ц}_{\text{пи}} + \text{О}_{\text{мри}}) \cdot \text{Н}_{\text{дс}}}{100\%}\\
  &= \frac{(5391624 + 218807) \cdot 20\%}{100\%} = 1009878 \text{руб},
\end{align*}
где \( \text{Н}_{\text{дс}} \) --- норматив НДС.

Прогнозируемая отпускная цена:
\begin{align*}
  \text{Ц}_{\text{ои}} &= \text{Ц}_{\text{пи}} \cdot (1 + \text{Н}_{\text{пр}}) + \text{НДС}_{\text{и}}=\\
  &= 5391624 \cdot 1,24 + 1009878 = 7914299 \text{руб},
\end{align*}
где \( \text{Н}_{\text{пр}} \) --- налог на прибыль.

\emph{Затраты на освоение ПО}. Затраты на освоение определяются по нормативу (\( \text{Н}_{\text{о}} = 10\% \)) от себестоимости ПО в расчете на 3 месяца и рассчитываются по формуле
\begin{displaymath}
  \text{Р}_{\text{ои}} = \frac{\text{С}_{\text{пи}} \cdot \text{Н}_{\text{о}}}{100\%} = \frac{3851160 \cdot 10\%}{100\%} = 385116 \text{руб}.
\end{displaymath}

\emph{Затраты на сопровождение ПО}. Затраты на сопровождение определяются по установленному нормативу (\( \text{Н}_{\text{с}} = 20\% \)) от себестоимости ПО (в расчете на год) и рассчитываются по формуле
\begin{displaymath}
  \text{Р}_{\text{си}} = \frac{\text{С}_{\text{пи}} \cdot \text{Н}_{\text{с}}}{100\%} = \frac{3851160 \cdot 20\%}{100\%} = 770232 \text{руб}.
\end{displaymath}

Заказчик оплачивает организации-разработчику всю сумму расходов по проекту, включая прибыль. После уплаты налогов из прибыли в распоряжении заказчика остается чистая прибыль от проекта. Ввиду того что ПО разрабатывалось для одного объекта, чистую прибыль можно считать в качестве экономического эффекта организации-разработчика от реализованного проекта.

\subsection{Расчет экономического эффекта у пользователя}
Экономический эффект у пользователя выражается в экономии трудовых, материальных и финансовых ресурсов, которая в конечном итоге также через уровень затрат, цену и объем продаж выступает в виде роста ЧД или ЧДД пользователя.

Для определения экономического эффекта от использования нового ПО у пользователя необходимо сравнить расходы по всем основным статьям сметы затрат на эксплуатацию нового ПО (расходы на заработную плату с начислениями, затраты на расходные материалы, расходы на машинное время) с расходами по соответствующим статьям базового варианта.

{\footnotesize
  \tablecaption{Исходные данные для расчета экономии ресурсов в связи с применением нового ПО}
  \label{source-data}
  \tablefirsthead{\hline
    \multicolumn{1}{|>{\centering}p{0.265\textwidth}|}{\textbf{Наименование показателей}} &
    \multicolumn{1}{>{\centering}p{0.07\textwidth}|}{\textbf{Обо\-зна\-че\-ние}} &
    \multicolumn{1}{>{\centering}p{0.1\textwidth}|}{\textbf{Единицы измерения}} &
    \multicolumn{2}{>{\centering}p{0.22\textwidth}|}{\textbf{Объем функции (LOC)}} &
    \multicolumn{1}{>{\centering}p{0.2\textwidth}|}{\textbf{Наименование источника информации}}\\
    \hline
    & & &\multicolumn{1}{>{\centering}p{0.095\textwidth}|}{в базовом варианте} &
    \multicolumn{1}{>{\centering}p{0.095\textwidth}|}{в новом варианте} &\\
  }
  \tablehead{
    \multicolumn{6}{c}{{\normalsize Продолжение таблицы \thetable{}}}\\
    \hline
    \multicolumn{1}{|>{\centering}p{0.265\textwidth}|}{\textbf{Наименование показателей}} &
    \multicolumn{1}{>{\centering}p{0.07\textwidth}|}{\textbf{Обо\-зна\-че\-ние}} &
    \multicolumn{1}{>{\centering}p{0.1\textwidth}|}{\textbf{Единицы измерения}} &
    \multicolumn{2}{>{\centering}p{0.22\textwidth}|}{\textbf{Объем функции (LOC)}} &
    \multicolumn{1}{>{\centering}p{0.2\textwidth}|}{\textbf{Наименование источника информации}}\\
    \hline
    & & &\multicolumn{1}{>{\centering}p{0.095\textwidth}|}{в базовом варианте} &
    \multicolumn{1}{>{\centering}p{0.095\textwidth}|}{в новом варианте} &\\
  }
  \begin{xtabular}{|p{0.265\textwidth}|>{\centering}p{0.07\textwidth}|>{\centering}p{0.1\textwidth}|>{\centering}p{0.095\textwidth}|>{\centering}p{0.095\textwidth}|p{0.2\textwidth}|}
    \hline
    1. Капитальные вложения, включая затраты пользователя на приобретение ПО & $\text{К}_\text{пр}$ & руб. & & 7914299 & Договор заказчика с разработчиком\tabularnewline
    \hline
    2. Затраты на освоение ПО & $\text{К}_\text{ос}$ & руб. & & 385116 & Договор заказчика с разработчиком\tabularnewline
    \hline
    3. Затраты на сопровождение ПО & $\text{К}_\text{с}$ & руб. & & 770232 & Договор заказчика с разработчиком\tabularnewline
    \hline
    4. Затраты на укомплектование ВТ техническими средствами в связи с внедрением нового ПО & $\text{К}_\text{тс}$ & руб. & & 385116 & Сметы затрат на внедрение\tabularnewline
    \hline
    5. Затраты на пополнение оборотных средств в связи с экплуатацией нового ПО & $\text{К}_\text{об}$ & руб. & & 61668 & Смета затрат на внедрение\tabularnewline
    \hline
    6. Время простоя сервиса, обусловленное ПО, в день & $\text{П}_\text{1}, \text{П}_\text{2}$ & мин & 0 & 0 & Расчетные данные пользователя и паспорт ПО\tabularnewline
    \hline
    7. Стоимость одного часа простоя & $\text{С}_\text{п}$ & руб. & & 37955 & Расчетные данные пользователя и паспорт ПО\tabularnewline
    \hline
    8. Среднемесячная ЗП одного программиста & $\text{З}_\text{см}$ & руб. & & 816000 & Расчетные данные пользователя\tabularnewline
    \hline
    9. Коэффициент начислений на зарплату & $\text{К}_\text{нз}$ & & 1,5 & 1,5 & Рассчитывается по данным пользователя\tabularnewline
    \hline
    10. Среднемесячное количество рабочих дней & $\text{Д}_\text{р}$ & день & & 21,5 & Принято для расчета\tabularnewline
    \hline
    11. Количество типовых задач, решаемых за год & $\text{З}_\text{т1}, \text{З}_\text{т2}$ & задача & 18880 & 37760 & План пользователя 2011-2013\tabularnewline
    \hline
    12. Объем выполняемых работ & $\text{А}_\text{1}, \text{А}_\text{2}$ & задача & 18880 & 37760 & План пользователя\tabularnewline
    \hline
    13. Средняя трудоемкость работ в расчете на 1 задачу & $\text{Т}_\text{с1}, \text{Т}_\text{с2}$ & человеко-час & 0,1 & 0,05 & Рассчитывается по данных пользователя\tabularnewline
    \hline
    14. Количество часов на работы в день & $\text{Т}_\text{ч}$ & ч & 8 & 8 & Принято для расчета\tabularnewline
    \hline
    15. Ставка налога на прибыль & $\text{Н}_\text{п}$ & \% & & 24 & \tabularnewline
    \hline
  \end{xtabular}
}

Особое значение имеет оценка капитальных затрат на приобрететение и использование ПО. Общие капитальные вложения ($\text{К}_\text{о}$) заказчика (потребителя), связанные с приобретением, внедрением и использованем ПО, рассчитываются по формуле:
\begin{align*}
  \text{К}_\text{о} &= \text{К}_\text{пр} + \text{К}_\text{ос} + \text{К}_\text{с} + \text{К}_\text{тс} + \text{К}_\text{об} =\\
  &= 7914299 + 385116 + 770232 + 385116 + 61668 =\\
  &= 9516431 \text{руб.},
\end{align*}
где $\text{К}_\text{пр}$ --- затраты пользователя на приобретение ПО по отпускной цене у разработчика с учетом стоимости услуг по эксплуатации;
$\text{К}_\text{ос}$ --- затраты пользователя на освоение ПО;
$\text{К}_\text{с}$ --- затраты пользователя на оплату услуг по сопровождению ПО;
$\text{К}_\text{тс}$ --- затраты на доукомплектование ВТ техническими средствами в связи с внедрением ПО;
$\text{К}_\text{об}$ --- затраты на пополнение оборотных средств в связи с использованием нового ПО.

Экономия затрат на заработную плату ($\text{С}_\text{з}$) при использовании нового ПО в расчете на объем выполненных работ:
\begin{displaymath}
  \text{С}_\text{з} = \text{С}_\text{зе} \cdot \text{А}_\text{2} = 237 \cdot 37760 = 8949120 \text{руб.},
\end{displaymath}
где $\text{С}_\text{зе}$ --- экономия затрат на заработную плату при решении задач с использованием нового ПО в расчете на 1 задачу (руб.); $\text{А}_\text{2}$ --- объем выполненных работ с использованием нового ПО (задач).

Экономия затрат на заработную плату в расчете на 1 задачу ($\text{С}_\text{зе}$):
\begin{align*}
  \text{С}_\text{зе} &= \frac{\text{З}_\text{см} \cdot (\text{Т}_\text{с1} - \text{Т}_\text{с2}) : \text{Т}_\text{ч}}{\text{Д}_\text{р}} =\\
  &= \frac{816000 \cdot (0,1 - 0,05) : 8}{21,5} = 237 \text{руб.},
\end{align*}
где $\text{З}_\text{см}$ --- среднемесячная заработная плата одного программиста; $\text{Т}_\text{с1}, \text{Т}_\text{с2}$ --- снижение трудоемкости работ в расчете на 1 задачу; $\text{Т}_\text{ч}$ --- количество часов работы в день; $\text{Д}_\text{р}$ --- среднемесячное количество рабочих дней.

Экономия с учетом начисления на зарплату ($\text{С}_\text{н}$):
\begin{displaymath}
  \text{С}_\text{н} = 1.5 \cdot \text{С}_\text{з} = 1.5 \cdot 8949120 = 13423680 \text{руб.}
\end{displaymath}

Экономия за счет сокращения простоев сервиса ($\text{С}_\text{с}$) рассчитывается по формуле:
\begin{displaymath}
  \text{С}_\text{с} = \frac{(\text{П}_\text{1} - \text{П}_\text{2}) \cdot \text{Д}_\text{рг} \cdot \text{С}_\text{п}}{60} = 0 \text{руб.},
\end{displaymath}
где $\text{Д}_\text{рг}$ --- плановый фонд работы сервиса.

Общая экономия текущих затрат, связанных с использованием нового ПО ($\text{С}_\text{о}$):
\begin{displaymath}
  \text{С}_\text{о} = \text{С}_\text{н} + \text{С}_\text{с} = 13423680 + 0 = 13423680 \text{руб.}
\end{displaymath}

Внедрение нового ПО позволит пользователю сэкономить на текущих затратах, т.е. практически получить на эту сумму дополнительную прибыль. Для пользователя в качестве экономического эффекта выступает лишь чистая прибыль --- дополнительная прибыль, остающаяся в его распоряжении ($\Delta\text{П}_{\text{ч}}$), которая определяется по формуле:
\begin{align*}
  \Delta\text{П}_{\text{ч}} &= \text{С}_\text{о} - \frac{\text{С}_\text{о} \cdot \text{Н}_\text{п}}{100\%} = 13423680 - \frac{13423680 \cdot 24\%}{100\%} =\\
  &= 10201997 \text{руб.},
\end{align*}
где $\text{Н}_\text{п}$ --- ставка налога на прибыль.

В процессе использования нового ПО чистая прибыль в конечном итоге возмещает капитальные затраты. Однако полученные при этом суммы результатов (прибыли) и затрат (капитальных вложение) по годам приводят к единому времени --- расчетному году (за расчетный год принят 2010-ый год) путем умножения результатов и затрат за каждый год на коэффициент дисконтирования $\alpha$.

\emph{Расчет нормы дисконта с использованием кумулятивного метода.} Применительно к рассматриваему проекту принято акцентрировать внимание на следующих факторах риска:
\begin{itemize}
  \item ставка процента;
  \item уровень инфляции;
  \item рост спроса;
  \item стабильность дохода.
\end{itemize}

Ввиду того что ставки процента в рублях пока не воспринимаются как устойчивые, в качестве безрисковой ставки принята ставка по валютным депозитам в Сбербанке в размере $5\%$. Возможное влияние непредвиденных обстоятельств на величину этой ставки оценено премией за риск в пределах $1\%$.

Согласно прогнозным данным компетентных источников уровень инфляции за период с 2010 по 2014 г. стабилизируется на отметке $3\%$. Средняя за расчетный период инфляция составит $3\%$. Следовательно, инфляционную премию к безрисковой ставке можно принять в размере $4\%$.

По данным маркетинговых исследований, проведенных консалтинговой фирмой, спрос на ПО начнет сокращаться к 2014 г. Возрастет риск падения спроса. Премия за риск падения спроса установлена в размере $1\%$.

В соответствии с результатами анализа финансовой реализуемости проекта в течение рассчитанного периода из-за изменения спроса на ПО и колебания цен в отдельные периоды возможно снижение дохода от проекта. Премия за риск изменения дохода устанавливается в размере $1\%$.

Нормативная ставка дисконта составит:
\begin{displaymath}
  E = 0,05 + 0,04 + 0,01 + 0,01 + 0,01 = 0,12.
\end{displaymath}

Коэффициенты дисконтирования $\alpha_i$ рассчитываются по формуле:
\begin{displaymath}
  \alpha_t = (1 + E)^{t_p - t},
\end{displaymath}
где $E$ --- норматив приведения разновременных затрат и результатов, в долях едицицы или в процентах в год; $t_p$ --- расчетный период; $t$ --- период, потоки которого приводятся к расчетному.

\tablefirsthead{}
\tablehead{}
\begin{center}
  \begin{xtabular} {lc@{\hspace{1cm}}l}
    2010 г.: & $t = 0$ & $\alpha_1 = 1$;\tabularnewline
    2011 г.: & $t = 1$ & $\alpha_2 = 0,8929$;\tabularnewline
    2012 г.: & $t = 2$ & $\alpha_3 = 0.7972$;\tabularnewline
    2013 г.: & $t = 3$ & $\alpha_4 = 0,7118$;\tabularnewline
    2014 г.: & $t = 4$ & $\alpha_5 = 0,6355$.
  \end{xtabular}
\end{center}

Все рассчитанные данные экономического эффекта сводятся в таблицу:
{\footnotesize
  \tablecaption{Расчет экономического эффекта от использования ПО}
  \label{economics-effect}
  \tablefirsthead{
    \hline
    \multicolumn{1}{|>{\centering}p{0.3\textwidth}|}{\textbf{Показатели}} &
    \multicolumn{1}{>{\centering}p{0.15\textwidth}|}{\textbf{Единица измерения}} &
    \multicolumn{4}{>{\centering}p{0.4\textwidth}|}{\textbf{Годы}}\\
    \hline
    & & \multicolumn{1}{>{\centering}p{0.085\textwidth}|}{2010} &
    \multicolumn{1}{>{\centering}p{0.085\textwidth}|}{2011} &
    \multicolumn{1}{>{\centering}p{0.085\textwidth}|}{2012} &
    \multicolumn{1}{>{\centering}p{0.085\textwidth}|}{2013}\\
  }
  \tablehead{
    \multicolumn{6}{c}{{\normalsize Продолжение таблицы \thetable{}}}\\
    \hline
    \multicolumn{1}{|>{\centering}p{0.3\textwidth}|}{\textbf{Показатели}} &
    \multicolumn{1}{>{\centering}p{0.15\textwidth}|}{\textbf{Единица измерения}} &
    \multicolumn{4}{>{\centering}p{0.4\textwidth}|}{\textbf{Годы}}\\
    \hline
    & & \multicolumn{1}{>{\centering}p{0.085\textwidth}|}{2010} &
    \multicolumn{1}{>{\centering}p{0.085\textwidth}|}{2011} &
    \multicolumn{1}{>{\centering}p{0.085\textwidth}|}{2012} &
    \multicolumn{1}{>{\centering}p{0.085\textwidth}|}{2013}\\
  }
  \begin{xtabular}{|p{0.3\textwidth}|>{\centering}p{0.15\textwidth}|c|c|c|c|}
    \hline
    \emph{Результаты:} & & & & & \\
    \hline
    Прирост прибыли за счет экономии затрат ($\text{П}_\text{ч}$) & руб. & & $10201997$ & $10201997$ & $10201997$\\
    \hline
    То же с учетом файтора времени & руб. & & $8871657$ & $7713730$ & $6707814$ \\
    \hline
    \emph{Затраты:} & & & & & \\
    \hline
    Приобретение ПО ($\text{К}_\text{пр}$) & руб. & $7914299$ & & & \\
    \hline
    Освоение ПО ($\text{К}_\text{ос}$) & руб. & $385116$ & & & \\
    \hline
    Сопровождение ПО ($\text{К}_\text{с}$) & руб. & $770232$ & & & \\
    \hline
    Доукомплектование ВТ техническами средствами ($\text{К}_\text{тс}$) & руб. & $385116$ & & & \\
    \hline
    Пополнение оборотных средств ($\text{К}_\text{об}$) & руб. & $61668$ & & & \\
    \hline
    \emph{Всего затрат:} & руб. & $9516431$ & & & \\
    \hline
    То же с учетом фактора времени & руб. & $9516431$ & & & \\
    \hline
    \emph{Экономический эффект:} & & & & & \\
    \hline
    Превышение результата над затратами & руб. & $-9516431$ & $8871657$ & $7713730$ & $6707814$ \\
    \hline
    То же с нарастающим итогом & руб. & $-9516431$ & $-644774$ & $7068956$ & $13776770$ \\
    \hline
    Коэффициент приведения & единиц & $1$ & $0,8929$ & $0.7972$ & $0,7118$ \\
    \hline
  \end{xtabular}
}\\[0.5\baselineskip]

Реализация проекта ПО позволит заказчику снизить трудоемкость решения задач и сократить простои сервиса. Все затраты заказчика окупятся во втором году эксплуатации ПО. Проект представляется эффективным и полезным для заказчика.

\newpage
