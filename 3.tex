\section{Проектирование программного комплекса}

Важным этапом разработки любого программного комплекса является его проектирование. Любой проект, связанный с созданием программного продукта, требует предварительного проектирования, построения структуры и планирования сроков разработки. Лишь после предварительного утверждения плана разработки и выбора используемых технологий и структуры программного продукта начинается его реализация.

В данном разеделе будет произведен анализ основных требований; рассмотрены этапы проектирования основных модулей системы.

\subsection{Анализ требований}

Разработанный программный прогаммный комплекс по управлению централизованными продажами в системе Bycard должен предоставить единый интерфейс для покупки билетов на мероприятия. Это значительно облегчит разработку клиентов для продажи билетов: мобильных приложений, сайта.

Система должна аггрегировать актуальную информацию о проходящих мероприятиях: расписание, цены на билеты, наличие доступных для продажи мест, возможность покупки билета онлайн. А также предоставить информационными ресурсам удобный механизм интерграции, благодаря которому будет возможность размещать актуальную афишу мероприятий на сторонних сайтах, показывать кнопки для покупки билета.

Большое значение имеют затраты на подключение к системе группы новых объектов, синхронизации данных между объектами. Необходимо разработать механизм, который позволит подключать к системе объекты для продажи билетов с минимальными затратами.

\subsection{Основные модули системы}

\subsubsection{}


