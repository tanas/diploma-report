\section{Используемые технологии}

Выбор технологий является важным предварительным этапом разработки сложных информационных систем. Платформа и язык программирования, на котором будет реализована система, заслуживает большого внимания, так как исследования показали, что выбор языка программирования влияет на производительность труда программистов и качество создаваемого ими кода. Ниже перечислены некоторые факторы, повлиявшие на выбор технологий:
\begin{itemize}
    \item разрабатываемое ПО должно работать на операционной системе Debian используемую в качестве серверной ОС;
    \item дальнейшей поддержкой проекта, возможно, будут заниматься разработчики, не принимавшие участие в выпуске первой версии;
    \item имеющийся разработчик имеет опыт работы с объекто-ориентированными и с функциональными языками программирования.
\end{itemize}

Основываясь на опыте работы имеющихся программистов разрабатывать ПО целесообразно на объектно-ориентированном языке PHP. Приняв во внимание необходимость обеспечения доступности дальнейшей поддержки ПО, возможно, другой командой программистов, целесообразно не использовать малоизвестные и сложные языки программирования.

\subsection{Язык программирования PHP}

PHP — скриптовый язык программирования общего назначения, интенсивно применяемый для разработки веб-приложений. В настоящее время поддерживается подавляющим большинством хостинг-провайдеров и является одним из лидеров среди языков программирования, применяющихся для создания динамических веб-сайтов.

Язык и его интерпретатор разрабатываются группой энтузиастов в рамках проекта с открытым кодом. Проект распространяется под собственной лицензией, несовместимой с GNU GPL.

В области программирования для сети Интернет, PHP — один из популярных сценарных языков (наряду с JSP, Perl и языками, используемыми в ASP.NET) благодаря своей простоте, скорости выполнения, богатой функциональности, кроссплатформенности и распространению исходных кодов на основе лицензии PHP.

Популярность в области построения веб-сайтов определяется наличием большого набора встроенных средств для разработки веб-приложений[8]. Основные из них:
\begin{itemize}
    \item автоматическое извлечение POST и GET-параметров, а также переменных окружения веб-сервера в предопределённые массивы;
    \item взаимодействие с большим количеством различных систем управления базами данных (MySQL, MySQLi, SQLite, PostgreSQL, Oracle (OCI8), Oracle, Microsoft SQL Server, Sybase, ODBC, mSQL, IBM DB2, Cloudscape и Apache Derby, Informix, Ovrimos SQL, Lotus Notes, DB++, DBM, dBase, DBX, FrontBase, FilePro, Ingres II, SESAM, Firebird / InterBase, Paradox File Access, MaxDB, Интерфейс PDO);
    \item автоматизированная отправка HTTP-заголовков;
    \item работа с HTTP-авторизацией;
    \item работа с cookies и сессиями;
    \item работа с локальными и удалёнными файлами, сокетами;
    \item обработка файлов, загружаемых на сервер;
    \item работа с XForms.
\end{itemize}

В настоящее время PHP используется сотнями тысяч разработчиков. Согласно рейтингу корпорации TIOBE, базирующемся на данных поисковых систем, в июне 2013 года PHP находился на 5 месте среди языков программирования. К крупнейшим сайтам, использующим PHP, относятся Facebook, Wikipedia и др.


\subsection{Фреймворк Symfony}


Symfony — свободный каркас, написанный на PHP5, который использует паттерн Model-View-Controller.

Symfony предлагает быструю разработку и управление веб-приложениями, позволяет легко решать рутинные задачи веб-программиста. Работает только с PHP 5 (>=5.2.4 и желательно не 5.2.9 для Symfony 1.4, >=5.3.2 для Symfony 2). Имеет поддержку множества баз данных (MySQL, PostgreSQL, SQLite или любая другая PDO-совместимая СУБД). Информация о реляционной базе данных в проекте должна быть связана с объектной моделью. Это можно сделать при помощи ORM инструмента. Symfony поставляется с двумя из них: Propel и Doctrine.

Symfony бесплатен и публикуется под лицензией MIT. Проект спонсируется французской компанией Sensio.

\subsection{Паттерн  Model-View-Controller}

Model-view-controller (MVC, «модель-представление-поведение», «модель-представление-контроллер», «модель-вид-контроллер») — схема использования нескольких шаблонов проектирования, с помощью которых модель данных приложения, пользовательский интерфейс и взаимодействие с пользователем разделены на три отдельных компонента таким образом, чтобы модификация одного из компонентов оказывала минимальное воздействие на остальные. Данная схема проектирования часто используется для построения архитектурного каркаса, когда переходят от теории к реализации в конкретной предметной области.

Основная цель применения этой концепции состоит в разделении бизнес-логики (модели) от её визуализации (представления, вида). За счет такого разделения повышается возможность повторного использования. Наиболее полезно применение данной концепции в тех случаях, когда пользователь должен видеть те же самые данные одновременно в различных контекстах и/или с различных точек зрения. В частности, выполняются следующие задачи:
\begin{enumerate}
    \item к одной модели можно присоединить несколько видов, при этом не затрагивая реализацию модели. Например, некоторые данные могут быть одновременно представлены в виде электронной таблицы, гистограммы и круговой диаграммы;
    \item не затрагивая реализацию видов, можно изменить реакции на действия пользователя (нажатие мышью на кнопке, ввод данных), для этого достаточно использовать другой контроллер;
    \item ряд разработчиков специализируется только в одной из областей: либо разрабатывают графический интерфейс, либо разрабатывают бизнес-логику. Поэтому возможно добиться того, что программисты, занимающиеся разработкой бизнес-логики (модели), вообще не будут осведомлены о том, какое представление будет использоваться;
\end{enumerate}

Концепция MVC позволяет разделить данные, представление и обработку действий пользователя на три отдельных компонента:
\begin{itemize}
    \item модель (англ. Model). Модель предоставляет знания: данные и методы работы с этими данными, реагирует на запросы, изменяя своё состояние. Не содержит информации, как эти знания можно визуализировать.
    \item представление, вид (англ. View). Отвечает за отображение информации (визуализацию). Часто в качестве представления выступает форма (окно) с графическими элементами.
    \item контроллер (англ. Controller). Обеспечивает связь между пользователем и системой: контролирует ввод данных пользователем и использует модель и представление для реализации необходимой реакции.
\end{itemize}

Важно отметить, что как представление, так и контроллер зависят от модели. Однако модель не зависит ни от представления, ни от контроллера. Тем самым достигается назначение такого разделения: оно позволяет строить модель независимо от визуального представления, а также создавать несколько различных представлений для одной модели.

Для реализации схемы Model-View-Controller используется достаточно большое число шаблонов проектирования (в зависимости от сложности архитектурного решения), основные из которых «наблюдатель», «стратегия», «компоновщик».

Наиболее типичная реализация отделяет вид от модели путем установления между ними протокола взаимодействия, используя аппарат событий (подписка/оповещение). При каждом изменении внутренних данных в модели она оповещает все зависящие от неё представления, и представление обновляется. Для этого используется шаблон «наблюдатель». При обработке реакции пользователя вид выбирает, в зависимости от нужной реакции, нужный контроллер, который обеспечит ту или иную связь с моделью. Для этого используется шаблон «стратегия», или вместо этого может быть модификация с использованием шаблона «команда». А для возможности однотипного обращения с подобъектами сложно-составного иерархического вида может использоваться шаблон «компоновщик». Кроме того, могут использоваться и другие шаблоны проектирования, например, «фабричный метод», который позволит задать по умолчанию тип контроллера для соответствующего вида.

\subsection{Реляционная СУБД MySQL}

MySQL — свободная реляционная система управления базами данных. Разработку и поддержку MySQL осуществляет корпорация Oracle, получившая права на торговую марку вместе с поглощённой Sun Microsystems, которая ранее приобрела шведскую компанию MySQL AB. Продукт распространяется как под GNU General Public License, так и под собственной коммерческой лицензией. Помимо этого, разработчики создают функциональность по заказу лицензионных пользователей. Именно благодаря такому заказу почти в самых ранних версиях появился механизм репликации.MySQL является решением для малых и средних приложений. Входит в состав серверов WAMP, AppServ, LAMP и в портативные сборки серверов Денвер, XAMPP. Обычно MySQL используется в качестве сервера, к которому обращаются локальные или удалённые клиенты, однако в дистрибутив входит библиотека внутреннего сервера, позволяющая включать MySQL в автономные программы.

Гибкость СУБД MySQL обеспечивается поддержкой большого количества типов таблиц: пользователи могут выбрать как таблицы типа MyISAM, поддерживающие полнотекстовый поиск, так и таблицы InnoDB, поддерживающие транзакции на уровне отдельных записей. Более того, СУБД MySQL поставляется со специальным типом таблиц EXAMPLE, демонстрирующим принципы создания новых типов таблиц. Благодаря открытой архитектуре и GPL-лицензированию, в СУБД MySQL постоянно появляются новые типы таблиц.

\newpage
